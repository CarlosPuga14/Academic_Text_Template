\documentclass{report}
%-----------------------------------------------------
%                       PACKAGES
% ----------------------------------------------------
\usepackage[T1]{fontenc}
\usepackage[latin9]{inputenc}
\usepackage{fancyhdr}
\usepackage{geometry}
\usepackage[colorlinks=true, linkcolor=blue,citecolor=blue, urlcolor=blue]{hyperref}
\usepackage{indentfirst}
\usepackage{graphicx}
\usepackage{float}
\usepackage{setspace}
\usepackage{amsmath}
\usepackage{amssymb}
\usepackage{multirow}
\usepackage[table,xcdraw]{xcolor}
\usepackage{colortbl}
\usepackage{tabularx,booktabs}
\usepackage{enumerate}
\usepackage{titlesec}
\usepackage{caption}
\usepackage{subcaption}
\usepackage{nicefrac, xfrac}
\usepackage{subfiles}
\usepackage{mathtools}
\usepackage{bm}
\usepackage{tikz}
\usepackage[mode=buildnew]{standalone}
\usepackage{systeme}
\usepackage{stmaryrd}
\usepackage{import}
\usepackage[english]{babel} 
\usepackage{times}
\usepackage{pgfplots}
\usepackage{makecell}
\usepackage{threeparttable}
\usepackage{ragged2e}
\usepackage{tocloft}
\usepackage{booktabs}

%-----------------------------------------------------
%               DOCUMENT SETTINGS
% ----------------------------------------------------
\geometry{a4paper,top=30mm,bottom=20mm,left=30mm,right=20mm}

\pgfplotsset{compat=1.17} 

% Defining spaces between lines
\setstretch{1.5} 

\graphicspath{{Figures/}}

\tikzset{
	font={\fontsize{11pt}{12}\selectfont}}
%-----------------------------------------------------
%              PRE TEXTUAL ELEMENTS
% ----------------------------------------------------
\newenvironment{epigrafe}{\newpage\mbox{}\vfill\hfill\begin{minipage}[t]{0.5\textwidth}}
{\end{minipage}\newpage}

%-----------------------------------------------------
%               DOCUMENT INFORMATION
% ----------------------------------------------------
\begin{document}

%-----------------------------------------------------
%               PRE TEXTUAL PAGES
% ----------------------------------------------------
% Could not find another way to remove the page number from the first pages
\fancypagestyle{plain}{
    \fancyhf{}% Limpa todos os campos
    \fancyfoot[C]{}%
    \renewcommand{\headrulewidth}{0pt}%
}

\newpage
\def\logos{
    \noindent
    \raisebox{-.5\height}{\includegraphics[width=2.2cm]{Figures/logo-unicamp.pdf}}

    \vspace*{2cm}
    
    \noindent
    \begin{center} \large
        \MakeUppercase{\Uni}\\
        \Fac\\
    \end{center}
}

\def\openningpage{
  \logos
  \vskip 35mm
  \begin{center}
    \Large
    {\bf \@author}
    \vskip 25mm
      {\bf \@title}
      \vskip 25mm
    {\bf \titulo}
    \vfill
    \large
    Campinas\\2024
  \end{center}
}

\openningpage % Cover page

\begin{epigrafe}
    \thispagestyle{empty}
    \centering
    \textit{``Dois problemas se misturam: \\ a verdade do universo e a presta\c{c}\~{a}o que vai vencer.''}
    \begin{flushright}
       - Raul Seixas
    \end{flushright}
\end{epigrafe} % First pages

\chapter*{Resumo}
    Este \'{e} o resumo em portugu\^{e}s do trabalho.


\chapter*{Abstract}
    This is the abstract in English of the work. % Abstract

{\newpage
\listoffigures} % List of figures

{\newpage
\listoftables} % List of tables

\newpage
\tableofcontents % Table of contents

\newpage
% from here on, the page number is shown
% Chapter first page settings
\fancypagestyle{plain}{
    \fancyhf{}% Limpa todos os campos
    \fancyhead[R]{\thepage}%
    \renewcommand{\headrulewidth}{0pt}%
}

\fancypagestyle{headings}{%
    \fancyhf{}% Limpa todos os campos
    \fancyhead[L]{\textsc{\nouppercase{\leftmark}}}
    \fancyhead[R]{\thepage}% Numero da página à direita
    \renewcommand{\headrulewidth}{1pt}%
}

\pagestyle{headings}

%-----------------------------------------------------
%                    CHAPTERS
% ----------------------------------------------------
\chapter{sjnsa}
This is something
\begin{equation}
    a = 1,
\end{equation} \label{eq:1}
\autoref{eq:1} represents something

\newpage
\section{aans}
\cite{Lima_Thesis}

%-----------------------------------------------------
%                    BIBLIOGRAPHY
% ----------------------------------------------------
\bibliographystyle{unsrt}
\addcontentsline{toc}{chapter}{Bibliography}
\bibliography{References}

\end{document}