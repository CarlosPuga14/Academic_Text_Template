\chapter{Examples}
% a good practice is to start each chapter with a summary
\lipsum[1-2]

\section{Equation}
This is something
\begin{equation}
    a = 1,
\end{equation} \label{eq:1}
Equation \eqref{eq:1} represents something

\section{Acronym and Glossary}
\acrlong{cfd}, which is abbreviated \acrshort{cfd}. 

This is a glossary example: \gls{a1}

\newpage
\section{Citation}

The information of \cite{Lima_Thesis} is pretty good. \textcite{arnold1984stable} has told that the information is pretty good. \textcite{Lima_Thesis}

\section{Figure}
\begin{figure}[h]
    \centering
    \includegraphics[]{Example.jpeg}
    \caption{This is a figure}
    \label{fig:1}
\end{figure}

Figure \ref{fig:1} is a good example of something

\section{Table}
% I think this sort of table is the prettiest way to do it
\begin{table}[h]
    \centering
    \caption{This is a table}
    \vspace*{2mm}
    \begin{tabular}{cc}
        \hline
        $Val_1$ & $Val_2$ \\
        \hline
        1 & 2 \\
        3 & 4 \\
        \hline
    \end{tabular}
    \label{tab:1}
\end{table}

Table \ref{tab:1} is a good example of something